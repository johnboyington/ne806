%This is a Tex Version of NE 806 Homework Template
%
\documentclass{amsart}
\setlength{\textheight}{9in}
\setlength{\topmargin}{-0.25in}
\setlength{\textwidth}{7in}
\setlength{\evensidemargin}{-0.25in}
\setlength{\oddsidemargin}{-0.25in}
\usepackage{amsfonts}
\usepackage[utf8]{inputenc}
\usepackage[T1]{fontenc}
\usepackage{graphicx} 
\usepackage[export]{adjustbox}
% needed to include these graphics
%\graphicspath{{./Pictures/}}      % only in case you want to keep the pictures in a separate
                                  % subdirectory; also see the appropriate line below
\usepackage{caption}
\usepackage{subcaption}
\usepackage{float}
\usepackage{framed}
\newcounter{temp}
\theoremstyle{definition}
\newtheorem{Thm}{Theorem}
\newtheorem{Prob}{Problem}
\newtheorem*{Def}{Definition}
\newtheorem*{Ans}{Answer}
\newcommand{\dis}{\displaystyle}
\newcommand{\dlim}{\dis\lim}
\newcommand{\dsum}{\dis\sum}
\newcommand{\dint}{\dis\int}
\newcommand{\ddint}{\dint\!\!\dint}
\newcommand{\dddint}{\dint\!\!\dint\!\!\dint}
\newcommand{\dt}{\text{d}t}
\newcommand{\dA}{\text{d}A}
\newcommand{\dV}{\text{d}V}
\newcommand{\dx}{\text{d}x}
\newcommand{\dy}{\text{d}y}
\newcommand{\dz}{\text{d}z}
\newcommand{\dw}{\text{d}w}
\newcommand{\du}{\text{d}u}
\newcommand{\dv}{\text{d}v}
\newcommand{\ds}{\text{d}s}
\newcommand{\dr}{\text{d}r}
\newcommand{\dth}{\text{d}\theta}
\newcommand{\bbR}{\mathbb{R}}
\newcommand{\bbN}{\mathbb{N}}
\newcommand{\bbQ}{\mathbb{Q}}
\newcommand{\bbZ}{\mathbb{Z}}
\newcommand{\bbC}{\mathbb{C}}
\newcommand{\dd}[2]{\dfrac{\text{d}#1}{\text{d}#2}}
\newcommand{\dydx}{\dfrac{\text{d}y}{\text{d}x}}
\renewcommand{\labelenumi}{{\normalfont \arabic{enumi}.}}
\renewcommand{\labelenumii}{{\normalfont \alph{enumii}.}}
\renewcommand{\labelenumiii}{{\normalfont \roman{enumiii}.}}
\font \bggbf cmbx18 scaled \magstep2
\font \bgbf cmbx10 scaled \magstep2
\usepackage{fancyhdr}
\usepackage{lipsum}
% Clear the header and footer
\fancyhead{}
\fancyfoot{}
% Set the right side of the footer to be the page number
\rfoot{\thepage}
\fancyhf{}
\pagestyle{fancy}

\begin{document}

\LARGE{NE-806: Neutronics}
 
\large
Homework Problem Set No. 2 due Thurs. October 4, 2018
 
Solutions by John Boyington \\
Note: Code used in this assignment is all located at https://github.com/johnboyington/ne806 and available for download/use.
\newline
\bigskip
 
% ----------------------------------------------------------------------------------------------------- 
\textbf{Problem 1:} Four parallel neutron beams have identical intensities of $3\times 10^6$ neutrons $cm^{-2} s^{-2}$ and intersect within a volume $V$ about the origin. The four beams are mono-energetic at the energy $E_0$. The first beam is in the x-direction, the second beam is in the -y-direction, the third beam is in the z-direction, and the fourth beam is in a direction given by
\begin{equation*}
    \hat{\Omega} = 0.5000 \hat{i_x} + 0.7071\hat{i_j} + 0.5000\hat{i_z}
\end{equation*}
(a) (5 points) What is the flux density at every point in the volume $V$?\newline
(b) (5 points) What is the current density at every point in the volume $V$?\newline
(c) (5 points) What is the magnitude of the current density at every point in $V$?\newline
(d) (5 points) What is the net number of neutrons crossing unit area in $V$ in the y-direction?\newline
 
\textbf{Solution}

(a) The flux density is simply 4 times the per-beam intensity: \\
\boxed{1.20 \times 10^7 cm^{-2} s^{-1}} \\

(b) Current density is obtained by summing the individual vector components of the beams: \\
\boxed{(4.50 \times 10^6) \hat{i}_x + (-8.79 \times 10^5) \hat{i}_y + (4.50 \times 10^6) \hat{i}_z cm^{-2} s^{-1}}

(c) Taking the 2-norm of the vector gives the magnitude: \\
\boxed{6.42 \times 10^6 cm^{-2} s^{-1}}

(d) Neutrons crossing the y-direction is the just the $y$ component of the current density vector: \\
\boxed{8.79 \times 10^5 cm^{-2} s^{-1}}



% -----------------------------------------------------------------------------------------------------
\newpage
\textbf{Problem 2:} You have determined that the angular flux density in a given problem can be expressed as
\begin{equation*}
    \psi(x,\mu) = 40,000 e^{-x/3} \bigg[ 2+ \sin(x \mu/3) \bigg] \qquad \textnormal{for} \qquad 0 \leq x \leq 10, \qquad -1 \leq \mu \leq 1
\end{equation*}
(a) (10 points) Use Monte Carlo to estimate total flux density at $x=x_0=5$. \newline
(b) (10 points) Give the precision of your estimate.\newline
 
\textbf{Solution}

(a) Finding the flux density at $x=x_0=0.5$ is like integrating the equation over $\mu$ while holding $x$ constant.
The following equation is used to find an integral with Monte Carlo in this case:
$$ \int^a_b f(x) \approx \frac{b - a}{N} \sum^N_i f(x, \rho_i) $$

This equation results in the following solution: \\
\boxed{3.02 \times 10^4 cm^{-2} s^{-1}}

(b)
And the following equation gives the error in this integral:

$$ \sigma \approx \frac{1}{\sqrt{N}} \frac{N}{N-1} \sqrt{\bar{x^2} - \bar{x}^2} $$

Applying this and using 1 million histories gives: \\
\boxed{5.49 \times 10^0 cm^{-2} s^{-1}}


% ----------------------------------------------------------------------------------------------------- 
\newpage
\textbf{Problem 3:} Lambert's law states that the angular intensity of the radiation leaving a surface varies as the cosine, $\omega$, of the angle between the outward normal and the radiation direction. Consider a medium that emits neutrons from the surface into a vacuum according to Lambert's law. \newline
(a) (5 points) What is the PDF for $\omega$?\newline
(b) (5 points) What is the CDF for $\omega$?\newline
(c) (5 points) What is the mean value of $\omega$?\newline
(d) (5 points) What fraction of neutrons are emitted at angles between 30$^\circ$ and 45$^\circ$?\newline
 
\textbf{Solution}

(a) The PDF is given by the following equation: \\
\boxed{ PDF = \cos{\theta} }

(b) And integrating from 0 to $\theta$ gives the CDF: \\
\boxed{ CDF = \sin{x} }

(c) Finding where the cdf is equal to 0.5 gives the mean $x$ value of a function. \\
$$ 0.5 = \sin{\theta} $$
Then plugging this into the original pdf gives the average value for $\omega$: \boxed{0.866}

(d) And using the CDF, with bounds at 30$^\circ$ and 45$^\circ$ gives the fraction \boxed{0.20711}


% ----------------------------------------------------------------------------------------------------- 
\newpage
\textbf{Problem 4:} (40 points) Consider a critical slab reactor of width $T=1$ mfp. Assume a quadrature has been applied to the NTE that leads to the eigenvalue equation
\begin{equation*}
    B\Phi = \lambda \Phi,
\end{equation*}
with $\lambda = \frac{2}{c}$ and $B$ given as follows \newline
\begin{table}[h!]
\begin{tabular}{llllllllll}
0.5109 & 0.1823 & 0.1223 & 0.0906 & 0.0702 & 0.056  & 0.0454 & 0.0374 & 0.0311 & 0.026  \\
0.1823 & 0.3555 & 0.1823 & 0.1223 & 0.0906 & 0.0702 & 0.056  & 0.0454 & 0.0374 & 0.0311 \\
0.1223 & 0.1823 & 0.3564 & 0.1823 & 0.1223 & 0.0906 & 0.0702 & 0.056  & 0.0454 & 0.0374 \\
0.0906 & 0.1223 & 0.1823 & 0.3568 & 0.1823 & 0.1223 & 0.0906 & 0.0702 & 0.056  & 0.0454 \\
0.0702 & 0.0906 & 0.1223 & 0.1823 & 0.357  & 0.1823 & 0.1223 & 0.0906 & 0.0702 & 0.056  \\
0.056  & 0.0702 & 0.0906 & 0.1223 & 0.1823 & 0.357  & 0.1823 & 0.1223 & 0.0906 & 0.0702 \\
0.0454 & 0.056  & 0.0702 & 0.0906 & 0.1223 & 0.1823 & 0.3568 & 0.1823 & 0.1223 & 0.0906 \\
0.0374 & 0.0454 & 0.056  & 0.0702 & 0.0906 & 0.1223 & 0.1823 & 0.3564 & 0.1823 & 0.1223 \\
0.0311 & 0.0374 & 0.0454 & 0.056  & 0.0702 & 0.0906 & 0.1223 & 0.1823 & 0.3555 & 0.1823 \\
0.026  & 0.0311 & 0.0374 & 0.0454 & 0.056  & 0.0702 & 0.0906 & 0.1223 & 0.1823 & 0.5109
\end{tabular}
\end{table}
 
Use the power method to estimate $c_{crit}$ to four decimal places. Show your work
 
\textbf{Solution}

First, set $\phi$ to some starting vector, in my case, a vector of randomized integers was used.
Then, given N iterations, the following is done:

$$ \phi_{k1} = B \phi_k $$
$$ \phi_{k1_n} = \phi_{k1} / MAX(\phi_{k1}) $$
$$ c = \frac{2}{\phi_{k1_n}} $$
$$ \phi_k = \frac{\phi_{k1}}{\phi_{k1_n}} $$

using an initial value of $c=1$.
This yielded the following approximation for $c_{crit}$:


\boxed{c_{crit} \approx 1.5861}
 
 
 
\end{document}
